\documentclass[12pt]{article}

\pagestyle{empty}
\setcounter{secnumdepth}{2}

\topmargin=0cm
\oddsidemargin=0cm
\textheight=22.0cm
\textwidth=16cm
\parindent=0cm
\parskip=0.15cm
\topskip=0truecm
\raggedbottom
\abovedisplayskip=3mm
\belowdisplayskip=3mm
\abovedisplayshortskip=0mm
\belowdisplayshortskip=2mm
\normalbaselineskip=12pt
\normalbaselines

\begin{document}

\vspace*{0.5in}
\centerline{\bf\Large Test Document}

\vspace*{0.5in}
\centerline{\bf\Large Team X}

\vspace*{0.5in}
\centerline{\bf\Large 03 Feb 2013}

\vspace*{1.5in}
\begin{table}[htbp]
\caption{Team}
\begin{center}
\begin{tabular}{|r | c|}
\hline
Name & ID Number \\
\hline\hline
X & Y \\
\hline
\end{tabular}
\end{center}
\end{table}

\clearpage

\section{Introduction}

{\it
An outline demonstrating Test cases and a few examples
 
}

\section{Test Plan}
Testing for successes and failures, conditions and combinations, for 4 to 10 tests per unit.  A few examples shown below.
  \subsection{Test Case   Unit Testing}

{\it
 Command/Range/Validation testing





 Testing to see if the test method itself is valid by default
}

 Example:

	public final void testIsValid()

	Command command= new Command();

	assertEquals("testIsValid", true, command.isValid());



  \subsection{Test Case 1.2  Intergration Testing}

 {\it Testing the Grid
}

{\it Upgrading/replacing a specific grid and checking if its coordinates matches.
}

Example:


	public final void testReplaceCellNamesByValue3()

	Command command= new Command();

	command.replaceCellNamesByValue("B4", 0.0);

	assertEquals("testIsAlphaNumeric","0.0",command.replaceCellNamesByValue("B4", 0.0));

Example:

{\it Checking if there exists the specific value within the specified cell row or column

}		public final void testGetCellRow3() 

		Grid grid = new Grid();

		assertEquals("Get Cell Row", 0, grid.getCellRow("D1"));

{\it 
 Comparing if it matches of an inserted value of a specified coordinate with a new value }

Example:


	public final void testGetCell0() 

	Grid grid = new Grid();

	grid.insertValue(2.0, 1, 1);

	assertEquals("Get Cell", 2.9, grid.getValueAt(1, 1));

{\it Testing the input of 2 particular cells }

  \subsection{Test Case 1.3  Intergration Testing}
{\it A value is inputted to a specific cell, and comparing it to a string input, namely A1+B1 }

Example:

	public final void testAlphanumericInput0() 

	Grid grid= new Grid();

	grid.insertValue(1.0, 0, 0);

	String equation = grid.alphanumericInput("A1 + B1");

	assertEquals("Test Alpha", "1.0 + 0.0", equation);



{\it Checking if the file is loaded }

Example:

	public void loadTest() 

Grid a = new Grid();

a.insertValue(79.1, 1, 2);

a.insertValue("B1+9", 1, 1);

Grid b = new Grid();

FileHandler fh = new FileHandler(new MainFrame("test"));

assertEquals("Load check", a, b);

 \section{Schedule}
After the duties of the developper were done, testing commenced, since 4 pm Feb 4 2013 until 11 pm.  Test cases are based on unit, intergration and system testing.



\subsection{Indicate which qualities (from requirements) were tested and which qualities were not tested.}
A grid of cells displaying the value of each cell has been tested.

An input line where formulas and primitive values for the selected cell are entered has been tested.

~~~~~A message line for displaying error messages has been tested, but not in test cases yet.~~~~~


\section{Functionality}

\subsection{Selecting a cell}
\subsection{enter a primitive value or formula for the selected cell }
\subsection{ automatic computation of cell values which are given by formulas} 
\subsection{ output the spreadsheet as a grid of cells with values }
\subsection{load and save spreadsheets to file} 
\subsection{quit}

\subsubsection{Unit X}

\section{Test Results}

{\it
List the tests, indicating which passed and which did not pass.
List requirements indicating the percentage of tests that passed for that requirement.
}

\section{References}

\appendix

\section{Description of Input Files}

Describe/include test data from input files.

\section{Description of Output Files}

Describe/include test expected output that are output files.

\end{document}
