\documentclass[12pt]{article}

\pagestyle{empty}
\setcounter{secnumdepth}{2}

\topmargin=0cm
\oddsidemargin=0cm
\textheight=22.0cm
\textwidth=16cm
\parindent=0cm
\parskip=0.15cm
\topskip=0truecm
\raggedbottom
\abovedisplayskip=3mm
\belowdisplayskip=3mm
\abovedisplayshortskip=0mm
\belowdisplayshortskip=2mm
\normalbaselineskip=12pt
\normalbaselines

\begin{document}

\section{Integration Testing}

\subsection{Portability Testing}

\subsubsection{Windows XP 32-Bit} \label{uc:1}

\noindent
{\bf Name}\\
Windows XP 32-BIT deployment

\noindent
{\bf Summary}\\
Port an executable version of FunSheets to a PC running Windows XP 32-BIT and test basic functionality and requirments.
\noindent\\
{\bf Precondition}\
\begin{enumerate}
\item Java project compiled without errors and all Major unit and subsystem tests have passed during development
\item Target PC has properly configured Java Run-Time-Enviornment (JRE) or Java Development Tool-kit (JDK) installed and working.
\end{enumerate}

\noindent
{\bf Main Scenario}\
\vspace*{-0.2in}
\begin{enumerate}
\item Using official version of FunSheets media (USB or DVD) install or copy the executable file to the target PC
\item Launch an instance of FunSheets by doubleclicking on the executable.
\item Testing various functionality according to Subsystem and Integration testing outlined in section XXX.XX
%Addison need to fill in the XXXXXXXXX
\end{enumerate}

{\bf Exceptions}\
\begin{enumerate}
\item None
\end{enumerate}
{\bf Expected Results}\
\begin{enumerate}
\item Subsystem test number XXX.XX to XXX.XX are successful.
\end{enumerate}

{\bf Postcondition And Results}\
\begin{enumerate}
\item Results are successful
\end{enumerate}

\noindent
{\bf Priority}\\
Successful
\noindent

\clearpage
%--------------------------------------------------------------------------------------------------
%                   Windows 7 test - 32-bit
%--------------------------------------------------------------------------------------------------
\subsubsection{Windows 7 32-Bit} \label{uc:1}

\noindent
{\bf Name}\\
Windows 7 32-BIT deployment

\noindent
{\bf Summary}\\
Port an executable version of FunSheets to a PC running Windows 7 32-BIT and test basic functionality and requirments.

\noindent
{\bf Precondition}\\
\begin{enumerate}
\item Java project compiled without errors and all Major unit and subsystem tests have passed during development
\item Target PC has properly configured Java Run-Time-Enviornment (JRE) or Java Development Tool-kit (JDK) installed and working.
\end{enumerate}

\noindent
{\bf Main Scenario}\
\vspace*{-0.2in}
\begin{enumerate}
\item Using official version of FunSheets media (USB or DVD) install or copy the executable file to the target PC
\item Launch an instance of FunSheets by doubleclicking on the executable.
\item Testing various functionality according to Subsystem and Integration testing outlined in section XXX.XX
%Addison need to fill in the XXXXXXXXX
\end{enumerate}

{\bf Exceptions}\
\begin{enumerate}
\item Windows 7 UAC (User Account Controls) can provide complications when running third-party software
\end{enumerate}
{\bf Expected Results}\
\begin{enumerate}
\item Subsystem test number XXX.XX to XXX.XX are successful.
\end{enumerate}

{\bf Postcondition And Results}\
\begin{enumerate}
\item In some cases UAC controls need to be set to allow FunSheets to run correctly. Firewall exceptions should be created.
\item Results are successful
\end{enumerate}

\noindent
{\bf Priority}\\
Successful
\noindent

\clearpage
%--------------------------------------------------------------------------------------------------
%                   Windows 7 test - 64-bit
%--------------------------------------------------------------------------------------------------
\subsubsection{Windows 7 64-Bit} \label{uc:1}

\noindent
{\bf Name}\\
Windows 7 64-BIT deployment

\noindent
{\bf Summary}\\
Port an executable version of FunSheets to a PC running Windows 7 64-BIT and test basic functionality and requirments.

\noindent
{\bf Precondition}\\
\begin{enumerate}
\item Java project compiled without errors and all Major unit and subsystem tests have passed during development
\item Target PC has properly configured Java Run-Time-Enviornment (JRE) or Java Development Tool-kit (JDK) installed and working.
\end{enumerate}

\noindent
{\bf Main Scenario}\
\vspace*{-0.2in}
\begin{enumerate}
\item Using official version of FunSheets media (USB or DVD) install or copy the executable file to the target PC
\item Launch an instance of FunSheets by doubleclicking on the executable.
\item Testing various functionality according to Subsystem and Integration testing outlined in section XXX.XX
%Addison need to fill in the XXXXXXXXX
\end{enumerate}

{\bf Exceptions}\
\begin{enumerate}
\item Windows 7 UAC (User Account Controls) can provide complications when running third-party software
\end{enumerate}

{\bf Expected Results}\
\begin{enumerate}
\item Subsystem test number XXX.XX to XXX.XX are successful.
\end{enumerate}

{\bf Postcondition And Results}\
\begin{enumerate}
\item In some cases UAC controls need to be set to allow FunSheets to run correctly. Firewall exceptions should be created.
\item Results are successful
\end{enumerate}

\noindent
{\bf Priority}\\
Successful
\noindent

\clearpage
%--------------------------------------------------------------------------------------------------
%                   Windows 8 test - 32-bit
%--------------------------------------------------------------------------------------------------
\subsubsection{Windows 8 32-Bit} \label{uc:1}

\noindent
{\bf Name}\\
Windows 8 32-BIT deployment

\noindent
{\bf Summary}\\
Port an executable version of FunSheets to a PC running Windows 8 32-BIT and test basic functionality and requirments.

\noindent
{\bf Precondition}\\
\begin{enumerate}
\item Java project compiled without errors and all Major unit and subsystem tests have passed during development
\item Target PC has properly configured Java Run-Time-Enviornment (JRE) or Java Development Tool-kit (JDK) installed and working.
\end{enumerate}

\noindent
{\bf Main Scenario}\
\vspace*{-0.2in}
\begin{enumerate}
\item Using official version of FunSheets media (USB or DVD) install or copy the executable file to the target PC
\item Launch an instance of FunSheets by doubleclicking on the executable.
\item Testing various functionality according to Subsystem and Integration testing outlined in section XXX.XX
%Addison need to fill in the XXXXXXXXX
\end{enumerate}

{\bf Exceptions}\
\begin{enumerate}
\item Windows 8 UAC (User Account Controls) can provide complications when running third-party software
\item FunSheets does not have an integrated Windows 8 Metro interface. Can only run from Desktop view
\end{enumerate}

{\bf Expected Results}\
\begin{enumerate}
\item Subsystem test number XXX.XX to XXX.XX are successful.
\end{enumerate}

{\bf Postcondition And Results}\
\begin{enumerate}
\item In some cases UAC controls need to be set to allow FunSheets to run correctly. Firewall exceptions should be created.
\item A Desktop icon should be created and used rather then Metro interface.
\item Results are successful
\end{enumerate}

\noindent
{\bf Priority}\\
Successful
\noindent

\clearpage
%--------------------------------------------------------------------------------------------------
%                   Windows 8 test - 64-bit
%--------------------------------------------------------------------------------------------------
\subsubsection{Windows 8 64-Bit} \label{uc:1}

\noindent
{\bf Name}\\
Windows 8 64-BIT deployment

\noindent
{\bf Summary}\\
Port an executable version of FunSheets to a PC running Windows 8 64-BIT and test basic functionality and requirments.

\noindent
{\bf Precondition}\\
\begin{enumerate}
\item Java project compiled without errors and all Major unit and subsystem tests have passed during development
\item Target PC has properly configured Java Run-Time-Enviornment (JRE) or Java Development Tool-kit (JDK) installed and working.
\end{enumerate}

\noindent
{\bf Main Scenario}\
\vspace*{-0.2in}
\begin{enumerate}
\item Using official version of FunSheets media (USB or DVD) install or copy the executable file to the target PC
\item Launch an instance of FunSheets by doubleclicking on the executable.
\item Testing various functionality according to Subsystem and Integration testing outlined in section XXX.XX
%Addison need to fill in the XXXXXXXXX
\end{enumerate}

{\bf Exceptions}\
\begin{enumerate}
\item Windows 8 UAC (User Account Controls) can provide complications when running third-party software
\item FunSheets does not have an integrated Windows 8 Metro interface. Can only run from Desktop view
\end{enumerate}

{\bf Expected Results}\
\begin{enumerate}
\item Subsystem test number XXX.XX to XXX.XX are successful.
\end{enumerate}

{\bf Postcondition And Results}\
\begin{enumerate}
\item In some cases UAC controls need to be set to allow FunSheets to run correctly. Firewall exceptions should be created.
\item A Desktop icon should be created and used rather then Metro interface.
\item Results are successful
\end{enumerate}

\noindent
{\bf Priority}\\
Successful
\noindent

\clearpage
%--------------------------------------------------------------------------------------------------
%                   Apple OS X
%--------------------------------------------------------------------------------------------------
\subsubsection{Apple OS X} \label{uc:1}

\noindent
{\bf Name}\\
Apple OS X deployment

\noindent
{\bf Summary}\\
Port an executable version of FunSheets to a PC running Apple OS X and test basic functionality and requirments.

\noindent
{\bf Precondition}\\
\begin{enumerate}
\item Java project compiled without errors and all Major unit and subsystem tests have passed during development
\item Target PC has properly configured Java Run-Time-Enviornment (JRE) or Java Development Tool-kit (JDK) installed and working.
\end{enumerate}

\noindent
{\bf Main Scenario}\
\vspace*{-0.2in}
\begin{enumerate}
\item Using official version of FunSheets media (USB or DVD) install or copy the executable file to the target PC
\item Launch an instance of FunSheets by doubleclicking on the executable.
\item Testing various functionality according to Subsystem and Integration testing outlined in section XXX.XX
%Addison need to fill in the XXXXXXXXX
\end{enumerate}

{\bf Exceptions}\
\begin{enumerate}
\item User account as approprate privilages to install software
\end{enumerate}

{\bf Expected Results}\
\begin{enumerate}
\item Subsystem test number XXX.XX to XXX.XX are successful.
\end{enumerate}

{\bf Postcondition And Results}\
\begin{enumerate}
\item Results are successful
\end{enumerate}

\noindent
{\bf Priority}\\
Successful
\noindent

\end{document}


