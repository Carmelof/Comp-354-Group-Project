\documentclass[12pt]{article}
\usepackage{graphicx}
\usepackage{hyperref}
\pagestyle{empty}
\setcounter{secnumdepth}{2}

\topmargin=0cm
\oddsidemargin=0cm
\textheight=22.0cm
\textwidth=16cm
\parindent=0cm
\parskip=0.15cm
\topskip=0truecm
\raggedbottom
\abovedisplayskip=3mm
\belowdisplayskip=3mm
\abovedisplayshortskip=0mm
\belowdisplayshortskip=2mm
\normalbaselineskip=12pt
\normalbaselines

\begin{document}

\vspace*{0.5in}
\centerline{\bf\Large FunSheets software}

\vspace*{0.5in}
\centerline{\bf\Large Team 2 - Iteration 2}

\vspace*{0.5in}
\centerline{\bf\Large 4 March 2013}

\vspace*{1.5in}
\begin{table}[htbp]
\caption{Team}
\begin{center}
\begin{tabular}{|r | c|}
\hline
Name & ID Number \\\hline\hline
Kevin Cameron & 9801448 \\\hline\hline
Addison Rodomista & 1967568 \\\hline\hline
Dragos Dinulescu & 6304826 \\\hline\hline
Adrian Max McCrea & 9801448 \\\hline\hline
Ghazal Zamani & 1971158 \\\hline\hline
Karim Kaidbey & 9654726 \\\hline\hline
Carmelo Fragapane & 6298265 \\\hline\hline
Long Wang & 9547967 \\\hline\hline
Simone Ovilma & 9112510 \\\hline\hline
Nicholas Constantinidis & 6330746 \\\hline\hline
Asmaa Alshaibi & 9738231 \\\hline
\end{tabular}
\end{center}
\end{table}

\clearpage

\section{Introduction}
[The instructions provided in blue are there to provide you indications describing the expected content of the respective sections. They are all to be deleted and replaced with appropriate content.]
[The introduction of the document provides an overview of the entire document, briefly introducing what are its goals, and what information is to be found in it.]

\section{Architectural Design}
[Updated and detailed version of the architecture design presented in deliverable 1, section 1.6. This section must give a high-level description of the system in terms of its modules and their respective purpose and exact interfaces.]

\subsection{Architecture Diagram}
[A UML class diagram or package diagram depicting the high-level structure of the system, accompanied by a one-paragraph text describing the rationale of this design, and possibly discussing the reasons for its discrepancies with the one presented in the first deliverable. It is mandatory that the system be divided into at least two subsystems, and that the purpose of each of these subsystems be exposed here.]

\subsection{Subsystem Interfaces Specifications}
[Specification of the software interfaces between the subsystems, i.e. specific messages (or function calls) that are exchanged by the subsystems. These are also often called “Module Interface Specifications”. Description of the parameters to be passed into these function calls in order to have a service fulfilled, including valid and invalid ranges of values. Each subsystem interface must be presented in a separate subsection.]  

\section{Detailed Design}
[Complete description of the system design, describing one subsystem separately in respective subsection. UML class diagrams are to be used, as well as a short textual description describing the purpose of each class.]  

\subsection{Subsystem \textless X\textgreater}
\subsubsection{Detailed Design Diagram}
[UML class diagram depicting the internal structure of the subsystem, accompanied by a paragraph of text describing the rationale of this design.]

\subsubsection{Units Description}
[List each class in this subsystem and write a short description of its purpose, as well as notes or reminders useful for the programmers who will implement them. List all attributes and functions of the class.]  

\section{Dynamic Design Scenarios}
[Describe some (at least two) important execution scenarios of the system using UML sequence diagrams. These scenarios must demonstrate how the various subsystems and units are interacting to achieve a system-level service. Units and subsystems depicted here must be compatible with the descriptions provided in section 2 and 3.]  

\end{document}
