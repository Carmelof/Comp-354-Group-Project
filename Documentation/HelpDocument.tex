\documentclass[12pt]{article}
\usepackage{graphicx}
\pagestyle{empty}
\setcounter{secnumdepth}{2}

\topmargin=0cm
\oddsidemargin=0cm
\textheight=22.0cm
\textwidth=16cm
\parindent=0cm
\parskip=0.15cm
\topskip=0truecm
\raggedbottom
\abovedisplayskip=3mm
\belowdisplayskip=3mm
\abovedisplayshortskip=0mm
\belowdisplayshortskip=2mm
\normalbaselineskip=12pt
\normalbaselines

\begin{document}

\vspace*{0.5in}
\centerline{\bf\Large Almost Excel}

\vspace*{0.5in}
\centerline{\bf\Large Team 2}

\vspace*{0.5in}
\centerline{\bf\Large 3 January 2013}

\vspace*{1.5in}
\begin{table}[htbp]
\caption{Team}
\begin{center}
\begin{tabular}{|r | c|}
\hline
Name & ID Number \\\hline\hline
Kevin Cameron & 9801448 \\\hline\hline
Addison Rodomista & 1967568 \\\hline\hline
Dragos Dinulescu & 6304826 \\\hline\hline
Adrian Max McCrea & 9801448 \\\hline\hline
Ghazal Zamani & 1971158 \\\hline\hline
x & y \\\hline\hline
Kevin Cameron & 9801448 \\\hline\hline
Kevin Cameron & 9801448 \\\hline\hline
Kevin Cameron & 9801448 \\\hline\hline
Kevin Cameron & 9801448 \\\hline\hline
Kevin Cameron & 9801448 \\\hline\hline
Kevin Cameron & 9801448 \\\hline
\end{tabular}
\end{center}
\end{table}

\clearpage

\section{System}
FunSheets is an electronic spreadsheet program used for storing, organizing and evaluating raw data. FunSheets is designed, written and distributed by Team 2 of COMP354 W/PP

\subsection{Purpose}
FunSheets is used to automate data orginization and the calculations of complex expressions involving large amounts of data.
\subsection{Context}
FunSheets will allow the user to create, open or modify a spread sheet and evaluate the source to produce a new spreadsheet featuring calculated value, evaluated expressions, and sorted information.
\section{Domain Concepts}

\section{Actors}
A list of actors (primary/secondary), and use cases are mentioned in this section.

\subsection{List of Actors}
\begin{enumerate}
\item Person as a primary actor.
\item System as a secondary actor.
\end{enumerate}

\subsection{Description of Actors}
\begin{enumerate}
\item Person:  A Person, is the main user of the application. Their roll is to input raw data or expressions manually or by opening existing files
\item System:  A software application that provides utilities for spreadsheet analysis and evaluation.
\end{enumerate}

\section{Use Cases}

\subsection{Overview}
Use Case model provides an understanding of the interaction between the user and the software
system. It consists of one or more actors and one or more use cases where actors represent particular
role and use cases represent actions performed by the system. 


\begin{figure}
\includegraphics{USE_CASE_DIAGRAM.jpg}
\caption{Use Case Diagram}
\label{fig:use-case-diagram}
\end{figure}

\clearpage

\subsubsection{Use Case 1} \label{uc:1}

\noindent
{\bf Name}\\
Select a Cell

\noindent
{\bf Summary}\\
A user will select a cell that can be used for various operations

\noindent
{\bf Actors}\\
1. Person\\
2. System\\

\noindent
{\bf Precondition}\\
The user has opened the program, and opened or created a spread sheet\\
\noindent
{\bf Main Scenario}\\
\vspace*{-0.2in}
\begin{enumerate}
\item User will click on the desired cell or enter the cell address.
\end{enumerate}

\noindent
{\bf Exceptions}\\
The cell is outside an acceptable range

\noindent
{\bf Postcondition}\\
Selecting a cell will trigger an event in the system to evaluate the input

\subsubsection{Use Case 2} \label{uc:2}

\clearpage

\section{Non-Functional Constraints}

\section{Data Dictionary}

\section{References}

\appendix

\section{Description of File Format: Tasks}

Describe input file format.

\section{Description of File Format: Persons}

Describe output file format.

\end{document}
